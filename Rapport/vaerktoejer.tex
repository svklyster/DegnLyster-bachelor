\documentclass[rapport.tex]{subfiles}

\begin{document}
\section{Værktøjer}
	\subsection{Udviklingsmiljø}
	Eye-tracking programmet er blevet udviklet delvist i Pythons IDE Idle, og delvist i Microsofts Visual Studio. Al kode er versionsstyret ved hjælp af versionsstyringssystemet Git.
	\subsection{OpenCV}
	OpenCV er et programbibliotek frit tilgængeligt under BSD licensen. Programbiblioteket tilbyder optimerede funktioner designet med henblik på hurtige udregninger med fokus på realtids-programmering. Desuden giver OpenCV en række funktioner for kommunikation med video-hardware, samt behandling af videofiler. OpenCV's funktioner er implementeret i kodesproget C++, men understøtter implementering igennem både Python, Java og Matlab, og kører på Windows, Linux, OS X, Android m.m. \cite{opencv}.
	\subsection{Python}
	Noget om python
	Numpy
	Tkinter er en Python-binding til GUI-værktøjet Tk. Dette interface giver mulighed for en nem implementering af det grafiske bruger-interface, og er frit tilgængeligt under en Python-licens. [reference til Tkinter]
	
	Find en reference der skitserer fordelene ved python over c++.
	\subsection{Diskussion}
	Har de valg vi har taget været relevante? Kort refleksion
	
		
\end{document}