\documentclass[rapport.tex]{subfiles}

\begin{document}
\section{Værktøjer}
	\subsection{Udviklingsmiljø}
	Eye-tracking programmet er blevet udviklet delvist i Pythons IDE Idle, og delvist i Microsofts Visual Studio. Al kode er versionsstyret ved hjælp af versionsstyringssystemet Git. Arbejde med UML, samt design af diverse diagrammer og figurer, er udført i Microsofts Visio. 
	\subsection{OpenCV}
	OpenCV er et programbibliotek frit tilgængeligt under BSD licensen. Programbiblioteket tilbyder optimerede funktioner designet med henblik på hurtige udregninger med fokus på realtids-programmering. Desuden giver OpenCV en række funktioner for kommunikation med video-hardware, samt behandling af videofiler. OpenCV's funktioner er implementeret i kodesproget C++, men understøtter implementering igennem både Python, Java og Matlab, og kører på Windows, Linux, OS X, Android m.m. \cite{opencv}.
	\subsection{Python}
	Python er et høj-niveau kodesprog der understøtter objekt-orienteret programmering, miksede høj-niveau datatyper, automatisk memory management, og et ekstensivt programbibliotek. 
	Python understøtter derfor hurtig implementering i forhold til både Java og C++. Desuden understøtter Python også wrapping med C/C++. Dette er basis for arbejde med OpenCV.  
	\subsubsection{Python vs. Java og C++} 
	
	Fordele \cite{PythonComp}: \begin{itemize}
		\item 3-5 gange kortere end Java og 5-10 gange kortere end C++. 
		\item Automatisk garbage collector. Ingen fokus på frigørelse af resourcer. 
		\item Miksede høj-niveau datatyper.
	\end{itemize}
	Ulemper: \begin{itemize}
		\item Langsommere end bade Java og C++, da datatyper skal ikke nødvendigvis er defineret før runtime. 
	\end{itemize}
	
	\subsubsection{Relevante programbiblioteker}
	Nogle af Python's benyttede programbiblioteker har haft stor indflydelse på projektet, og er derfor relevante at nævne: \\\\	
	\textbf{Numpy} er et højt optimeret programbibliotek til arbjede med numeriske operationer og tilbyder derfor en god understøttelse af array- og matrice-matematik, samt en ekstensiv række høj-niveau matematiske funktioner \cite{Numpy}. Numpy er derfor anvendt af OpenCV i Python. \\\\
	\textbf{Tkinter} er en Python-binding til GUI-værktøjet Tk. Dette interface giver mulighed for en nem implementering af det grafiske bruger-interface, og er frit tilgængeligt under en Python-licens \cite{Tkinter}.

	\subsection{Diskussion}
	Det valgte udviklingsmiljø har givet et sæt stærke værktøjer til design, implementering, debugging og profiling. Versionsstyring har været de naturlige valg når det kommer til programmering. 
	Brugen af Numpy har lagt grund for en nem omskrivning af Siboskas MATLAB-kode, da de fleste matematiske funktioner også findes i Numpy. 
		
\end{document}