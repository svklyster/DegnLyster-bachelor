\documentclass[rapport.tex]{subfiles}

\begin{document}

\section{Prolog}
	\subsection{Indledning}
	\subsection{Ordforklaring}
		\item \textbf{Session}: Dette term bliver brugt om en real-time eye-tracking måling foretaget af i programmet. Sessionen beskriver det enkelte målingsforløb fra start til slut (inklusiv pauser). Til hver session vil der være tilknyttet en seperat data-fil. 
		\item \textbf{Data-fil}: Dette er den fil der vil blive tilknyttet til hver session. Filen vil forventes at indeholde al relevant data i forbindelse med real-time eye-tracking måling. Filen vil blive kreeret af programmet og vil være tilgængelig til brugeren. 
		\item \textbf{Testperson}: Det er denne person der foretages real-time eye-tracking på. Personen er sammen med brugeren en del af kalibreringsrutinen. Denne person ses ikke som aktør i systemet. 
		\item \textbf{Trigger}: For at kunne holde en synkronisering imellem real-time eye-tracking softwaren og andre målinger (EEG), er der givet et trigger-signal. Dette signal består af en ændring af lys-intensitet. 
	\subsection{Projektformulering}
	Fra projektoplægget er føljende beskrivelse givet:\\
	\\
\textit{	Eye-tracking is widely used in different research areas as for example in psychology, in analysis of man-computer interactions, and in behavioural studies. Eye-tracking is also used in computer gaming. At ASE eye-tracking is used for research both in the biomedical lab and in the vision lab. The system as it is now is based on off-line Matlab processing of camera data. The purpose of this project is to design and implement a software system for real-time acquisition of the eye-movements. The basic principle of the eye-tracking system is based on reflection from two IR-LED’s from the eyes. By identifying the reflections, and doing some geometrical computations, it is possible to determine the users gaze vector. 
	The project will build on an existing hardware setup comprising a camera, IR-sources and a computer screen. The primary objective of the project is to design and build a flexible software system that can process the camera data in real-time, and output the gaze vector. The image processing will build on an existing Matlab implementation. The system must be designed such that there is a flexible interface to the image processing part in order to facilitate new image processing algorithms to be tested in the system. For students with particular interest in image processing, a secondary objective could be to further develop on the image processing algorithms.} 
	\\
	
	\subsection{OpenEyes og Siboska}
	Kort beskrivelse af de to algoritme-implementeringer projektet har taget udgangspunkt i.	
	
		
\end{document}