\documentclass[rapport.tex]{subfiles}

\begin{document}
	\section{Design}
	For at oprette en software arkitektur der overholder de krav stillet i kravspecifikationen, er der gjort brug af UML (Unified Modelling Language) \cite{UMLurl}. UML tillader en tilnærmelsesvis direkte omskrivning af krav opstillet som use cases, til UML-diagrammer der skitserer en software arkitektur. 
	Da der i kravspecifikationen er gjort brug af UML til udarbejdelse af de forskellige use cases, har det derfor været muligt af skrive sekvensdiagrammer ud fra hver enkelte use case. 
	
	Sideløbende med udviklingen af sekvensdiagrammerne er de forskellige klasser blevet forfattet. Følgende afsnit beskriver grundlæggende tanker og argumentation for valget af klasser.
	\subsection{Indledning}
	Kort om hvad dette afsnit indeholder, og hvorfor det er relevant. 	
	\subsection{Modularisering}
	Programmet er ønsket opbygget, så der af en udestående person, er mulighed for fremtidig ændringer i algoritmen. Programmet er derfor struktureret op omkring at have et fastlagt interface med konsistente input/outputs med en selvstændig algoritme-klasse. 
	\subsection{Applikationsarkitektur}
	Overvejelser om opdelingen af klasser. Hver klasse varetager en opgave med henblik på en af systemets grænseflader. GUI er varetaget af en klasse. De tre genererede filer log, session, og kalibrering er håndteret af tre klasser. Kommunikation med eksternt kamera er håndteret af sin egen klasse. Kommunikation med algoritmen har sin egen klasse. 
		\subsubsection{UserInterface}
		Denne klasse foretager al kommunikation med aktøren \textit{bruger} igennem et grafisk bruger-interface (GUI). For at have en konstant opdatering af interfacet, forventes denne klasse at blive afviklet i egen tråd. 
		\subsubsection{SessionHandler}
		Her håndteres al data tilknyttet opsætning af eye-tracking-session. 
		\subsubsection{LogHandler}
		Denne klasse håndterer eye-tracking-programmets log-fil. Data fra eye-tracking bliver her pakket og gemt i en log-fil. Kommunikation af målings-relevant data til UserInterface-klassen foretages af denne klasse.
		\subsubsection{VideoCapture}
		\subsubsection{EyeTrackingHandler}
		Denne klasse håndterer kommunikation og instantiering af klassen ETAlgorithm. 
		
	\subsection{Algoritme-arkitektur}
	Klassen ETAlgorithm er kernen af eye-tracking-systemet. Her foretages alle udregninger med henblik på eye-tracking. Denne klasse ønskes isoleret så meget som muligt fra resten af systemet, således at der altid kan foretages ændringer i klassens interne kode, uden at det påvirker resten af systemet. Konsistente grænseflader for denne klasse er derfor essentielt. 
	ETAlgorithm bliver håndteret af klassen EyeTracking, modtager et videosignal, og returnerer XY-koordinat samt trigger-niveau. 
	
	\subsection{Bruger-interface}
	Interfacet er designet med henblik på intuitivt brug, så ledes at det ikke er nødvendigt med dybere introduktion til programmet. Forskellige stadier af hvad man kan gøre i interfacet reducerer muligheden for uønskede handlinger. Det er for eksempel ikke muligt at starte eye-tracking før nødvendig opsætning er fuldført. 
	
	Følgende figurer beskriver det forventede grafiske bruger interface. Interfaces er designet ud fra use case diagrammerne beskrevet i kravspecifikationen.
	\subsection{Diskussion}
\end{document}