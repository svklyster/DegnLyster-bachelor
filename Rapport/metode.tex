\documentclass[rapport.tex]{subfiles}

\begin{document}
\section{Metode}
	\subsection{Metoder}
	Til dette projekt er der blevet benyttet V-modellen som udviklingmetode \cite{Vmodel}, og UML (Unified Modelling Language) \cite{UML} til design af softwarearkitekturen . V-modellen tilbyder en overskuelig tilgang til softwareudvikling i et projekt hvor en række krav er fastsat som udgangspunkt for udviklingen. Ved hjælp af UML er de forskellige krav omskrevet til use cases, og derfra er der udviklet design-diagrammer for softwarearkitekturen. 
	V-modellen følger i dette projekt følgende stadier fra analyse til implementering:
	\begin{itemize}
		\item Opstilling af krav
		\item Analyse
		\item Design
		\item Implementering
	\end{itemize}
	Disse stadier har hver endt ud i et dokument der beskriver overvejelser og arbejdet udført. \\
	
	Efter implementering af koden følger modellen en række test-stadier:
	\begin{itemize}
		\item Deltest
		\item Integrationstest
		\item Accepttest
	\end{itemize}
	
	V-modellens fleksibilitet tillader at man ved hvert teststadie kan gå tilbage og tilpasse analyse, design og implementering. 
	V-modellen giver derfor mulighed for at opstille en række krav, omskrive dem til softwarearkitektur, implementere arkitekturen, foretage tests på den implementerede kode, og derefter gå tilbage rette hvad der kunne være nødvendigt. 
	I et projekt af denne størrelse, hvor det forventes at der skal tilegnes ny viden, giver denne model derfor god mulighed for tidligt at støde på eventuelle problemer ved implementeringen, og derefter søge ny viden der kan benyttes til at opnå de opstillede krav. 
	
	\subsection{Planlægning}
	Hvad har der været gjort angående strukturering af projektet
	\subsection{Diskussion}
	Har de valg vi har taget været relevante? Kort reflektion
		
\end{document}