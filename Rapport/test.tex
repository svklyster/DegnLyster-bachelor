\documentclass[rapport.tex]{subfiles}

\begin{document}
\section{Test}
	\subsection{Indledning}
	\subsection{Deltest}
	Deltestene har til formål at teste hver enkelte klasses funktionalitet. 
	Da de forskellige klasser er skrevet i hver sin python-fil, og da python-filer uden problemer kan inkluderes i python-programmer, har hver klasse været mulig at teste ved at skrive kalde funktionaliteter med specifikke inputs og derefter læse returværdierne/observere hændelserne. \\
	
	Ved at skrive en test-bench (her et python-program) har det været muligt at teste hver klasses funktionaliteter. Alle test-cases er udført i værktøjet Visual Studio, hvilket giver fri mulighed for at aflæse de forskellige variabler i runtime. \textit{Følgende er fremgangsmåde og resultat af de forskellige test-cases:}
	\subsection{Integrationstest}
	Integrationstesten har til formål at teste kommunikationen klasser imellem. 
	\subsection{Accepttest}
	Accepttesten er formuleret efter de opstillede krav i afsnit \ref{sec:kravspec}. Formålet ved accepttesten er at teste det samlede produkt op imod de stillede krav. Testen afsluttes når alle specificerede test cases er gennemført og godkendt. I tilfælde hvor et krav ikke har kunne fuldføres, er der udfærdiget en problemrapport årsagen til underkendelsen. 
	
	\subsection{Performance-evaluering}
	\subsection{Diskussion af testresultater}
		
\end{document}