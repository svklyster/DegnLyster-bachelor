\documentclass[rapport.tex]{subfiles}

\begin{document}
\section{Analyse}
	\subsection{Indledning}
	Formålet med analysedelen af projektet er at få en indledende indsigt i emnet. Det indebærer at få skabt et overblik over hvad der skal laves, hvorfor det skal laves, og hvilke krav der er til de ting som skal laves. Bemærk at disse krav ikke er endegyldige, og muligvis justeres igennem projektforløbet. Derudover er det også vigtigt at nævne at det der bliver diskuteret overvejende er koncepter eller teori. Konkret design og implementering vil blive gennemgået senere, i design og implementeringsdelene af rapporten.
	\subsection{Systemoversigt}
	
	Blokdiagram: Userskærm, testskærm, IR-LED, kamera.
	Blokdiagram2: Overordnet blokdiagram over software. (Figur 2, geometric approach to eyetracking)
	
	Billede af opstilling.
	
	
	
	
	Dette er opstillingen som projektet tager udgangspunkt i. Argumentationen for	valget af denne opstilling kan ses i indledningen til projektet (Ref indledning).
	
	\subsection{Funktionalitetskrav}
	
		\subsubsection{Use case eksempel - Start måling}
		\begin{tabular}{|l|p{7.7cm}|}
			\hline \textbf{Sektion} 	& \textbf{Kommentar} \\ 
			\hline Mål  & Programmet påbegynder real-time eye-tracking \\ 
			\hline Initiering  & Initieres af aktøren \textit{bruger} \\ 
			\hline Aktører & Aktøren \textit{bruger} og aktøren \textit{kamera} \\ 
			\hline Antal samtidige forekomster & 1 \\ 
			\hline Startbetingelser & Computerprogrammet skal være opstartet, \textit{kamera} skal være tændt, programmet skal være kalibreret. \\ 
			\hline Slutresultat – succes & Programmet har påbegyndt real-time eye-tracking\\ 
			\hline Slutresultat – undtagelse & Programmet alarmerer \textit{bruger} at der ikke er foretaget kalibrering \\ 
			\hline Normal forløb & \begin{enumerate}
				\item \textit{Bruger} klikker på knappen ”Start”.
				\item Programmet starter ny måling.
				\item Visuel feedback på GUI viser at måling er i gang.
			\end{enumerate} \\  
			\hline Undtagelsesforløb & Programmet kan ikke starte ny måling. Programmet melder
			at kalibrering ikke er foretaget.\\
			\hline 
		\end{tabular}
		
	\subsection{Overvejelser, Funktionelle krav}	
	
	\subsubsection{Real-time eye-tracking}
	Dette krav er stillet af udbyderen af projektet, og kan
	betragtes som et af de mest fundamentale krav i projektet. Projektet udspringer fra et tidligere projekt, hvor billedbehandlingen af kameradata foregik offline. Det skal optimeres i en tilstrækkelig grad til at kunne køre i realtime i stedet for, med en framerate på 100 billeder per sekund.
	
	
	\subsubsection{Kalibrering}
	For at systemet skal fungere korrekt skal der først foretages en kalibrering.
	Målinger kan foretages uden kalibrering, men det vil ikke være muligt at oversætte de data til et sæt skærmkoordinater. Denne kalibrering er derfor essentiel for systemet.
	
	
	\subsubsection{Output}
	Resultater fra målinger skal gemmes i en log-fil tilgængelig til brugeren. Dette krav stammer fra et ønske fra udbyder om at have adgang til data efter behandling. Der vil i
	design fasen af projektet blive fremsat en protokol, der beskriver hvordan data skal gemmes i log-filen. 
	
	
	\subsubsection{Brugertilgang}
	Brugergrænsefladen behøver ikke nødvendigvis at være ret optimeret (Den er også lavet i python),
	men skal derimod opfylde nogle funktionelle krav, hvilket kan betragtes som en
	anden form for optimering, brugeroplevelsesoptimering. Den optimering er beregnet til at
	sørge for at applikationen bliver designet og implementeret på en måde som medfører en
	bedre oplevelse for brugeren, i form af en simpel brugergrænseflade, tilstrækkelig
	funktionalitet til at dække formodede behov, samt et design der gør det let at udvide
	applikationen på et senere tidspunkt.
	
	De resterende funktionelle krav er blevet uddybet ved hjælp af Use Cases -REF-.
	
	
	\subsection{Overvejelser, Ikke-funktionelle krav}
	
	\subsubsection{Fejlmargin}
	
	
	
	\subsubsection{Real-time}
	
	
	
	\subsubsection{Kodesprog}
	Det er blevet aftalt med projektudbyder at prototypen skal udarbejdes med en C++ backend og Python frontend.
	C++ er valgt som backend fordi det er et relativt hurtigt kodesprog, og fordi gruppen har tidligere erfaring
	med C++. Python blev foreslået til gruppen af projektvejleder, og idet Python har en udvidelse der tillader
	en let måde at have Python frontend samt C++ backend var det en oplagt mulighed at anvende Python til front
	end delen af prototypen.
	\subsection{Starburst-algoritmen}
	I det følgende vil koden som projektet tager udgangspunkt i blive gennemgået i dybden, hvilket også indebærer processeringstid for de enkelte subrutiner. I forlængelse af dette vil begrundelserne for krav der har med algoritmen at gøre også blive givet i en relevant kontekst.
	
	Bemærk at procestiden er opgivet som procentdel af samlet procestid, og at værdierne kun omfatter processen selv, og ikke medregner den tid der bruges på metoder der kaldes undervejs. Procestiden for disse metodekald står ud for de enkelte metoder i stedet.
	
	\subsection{Algoritme Oversigt}
	\begin{figure}[h]
		\centering
		\includegraphics[width=0.7\linewidth]{../Algoritmediagram.png}
		\caption[Systemdiagram]{Systemdiagram for Real-time eye-tracking}
		\label{fig:Systemdiagram}
	\end{figure}
	
	\subsubsection{Calculate pupil and gaze}
	
	Main funktion - 0.17%
	
	\subsubsection{Locate corneal reflection}
	
	Finder reflektionspunker - 0.3%
	
	\subsubsection{Starburst pupil contour detection}
	
	Starburst algoritme - 0.63%
	
	\subsubsection{Locate Edge Points}
	
	Find pupil kant punkter - 44.71%
	
	\subsubsection{Fit ellipse ransac}
	
	Tilpas en ellipse til punkterne - 11.7%
	
	\subsection{Fokuspunkter}
	I det følgende underafsnit vil der kort blive beskrevet hvilke dele af systemet der med fordel kan fokuseres på. Dette er baseret på antal gange rutinen bliver kørt, samt hvor lang tid det tager for processen at blive færdig. Begrundelsen for dette er at det højst sandsynligt er lettere at øge performance med en betydelig del hvis de dele der tager længst tid først bliver optimeret.
	
	\subsubsection{Kantdetektion}
	Den mest tidskrævende del af algoritmen er selve starburst-delen, altså den del hvor pupilkantpunkter findes. Udover at optimere koden ved at køre i C, er der også mulighed for at justere variabler for at få processen til at tage kortere tid. Dette vil dog højst sandsynligt ske på bekostning af resultaternes nøjagtighed/præcision, eller systemets overordnede stabilitet. Der vil altså formentligt være behov for en balancering af performance overfor kvalitet. Denne balancering vil foregå iterativt i løbet af implementeringsfasen af projektet.
	
	\subsubsection{Ellipse Tilpasning}
	Den næstmest tidskrævende del af algoritmen er ellipsetilpasningen. Her vil det igen være muligt at justere på variabler, for at balancere performance og kvalitet. Derudover skal der ses nærmere på undtagelsestilstande hvor algoritmen gentages mange gange, RANSAC iterations = 10000. Hvis disse undtagelsestilstande opsluger meget tid og sker ofte, kunne det have en markant effekt på performance for systemet.
	
	\subsubsection{Kegleparametre til Ellipseparametre}
	En mindre betydelig, men dog stadig mærkbar del af algoritmen er den del som oversætter kegleparametre til ellipseparametre. Da denne del af algoritmen kun udgør omkring 6 procent af den samlede procestid, er denne blot nævnt som en mulig kandidat for optimering hvis de primære dele begynder at være sammenlignelige i procestid. 
	
	\subsubsection{Kamera Input}
	En sidste ting der med fordel kan fokuseres på at forbedre er indlæsning af data fra kameraet. Idet der ikke sker meget kompression bør det meste af arbejdet bestå af overførsler i harddisken, men hvis det viser sig at være for tungt kan vi forsøge at gøre noget.
	\subsection{OpenEyes og Siboska}
	I det følgende vil simuleringer i forbindelse med starburst algoritmen blive gennemgået. Disse simuleringer er blevet udført i et forsøg på at danne en bedre indsigt i sammenhængen imellem performance og resultater. Idet en overgang til en anden platform end MATlab formentligt ikke forøger performance tilstrækkeligt, skal der i stedet undersøges hvorvidt det er muligt at tilpasse forskellige variabler til at opnå en kortere processeringstid, alt imens resultaterne forbliver gode.
	
	\subsection{Simulering}
	
	Simuleringerne er foretaget med det kode og videodata som Daniel Sibozka har udleveret til gruppen. I det første afsnit vises resultaterne, og i andet afsnit beskrives hvilke ændringer af variabler der afprøves, og derefter vises resultaterne af disse ændringer.
	
	\subsubsection{Extract gaze vector from video}
	
	Processeringstid, CPU = 2.4 GHz
	
	Figur over resultater.
	
	\subsubsection{Variabel Ændringer}
	
	Antal Rays
	Antal RANSAC iterationer
	
	
	\subsubsection{Resultater}
	
	Figur 
	\subsection{Diskussion}
		
\end{document}