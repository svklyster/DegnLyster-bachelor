\documentclass[rapport.tex]{subfiles}

\begin{document}
\section{Analyse}
	\subsection{Indledning}
	\subsection{Systemoversigt}
	\subsection{Funktionalitetskrav}
	
		\subsubsection{Use case eksempel - Start måling}
		\begin{tabular}{|l|p{7.7cm}|}
			\hline \textbf{Sektion} 	& \textbf{Kommentar} \\ 
			\hline Mål  & Programmet påbegynder real-time eye-tracking \\ 
			\hline Initiering  & Initieres af aktøren \textit{bruger} \\ 
			\hline Aktører & Aktøren \textit{bruger} og aktøren \textit{kamera} \\ 
			\hline Antal samtidige forekomster & 1 \\ 
			\hline Startbetingelser & Computerprogrammet skal være opstartet, \textit{kamera} skal være tændt, programmet skal være kalibreret. \\ 
			\hline Slutresultat – succes & Programmet har påbegyndt real-time eye-tracking\\ 
			\hline Slutresultat – undtagelse & Programmet alarmerer \textit{bruger} at der ikke er foretaget kalibrering \\ 
			\hline Normal forløb & \begin{enumerate}
				\item \textit{Bruger} klikker på knappen ”Start”.
				\item Programmet starter ny måling.
				\item Visuel feedback på GUI viser at måling er i gang.
			\end{enumerate} \\  
			\hline Undtagelsesforløb & Programmet kan ikke starte ny måling. Programmet melder
			at kalibrering ikke er foretaget.\\
			\hline 
		\end{tabular}
		
		
	\subsection{Overvejelser}
		\subsubsection{Krav}
		\subsubsection{Kodesprog}
	\subsection{Starburst-algoritmen}
	\subsection{OpenEyes og Siboska}
	\subsection{Diskussion}
		
\end{document}