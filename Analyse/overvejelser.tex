\documentclass[analyse.tex]{subfiles}
\begin{document}
\section{Overvejelser angående systemkrav}
I analyseforløbet er der desuden blevet fremstillet en række krav, i samarbejde med projekt
udbyder. Formålet med disse krav er at få stillet konkrete krav til systemet, som der så vil
blive lavet analyse på, og derefter vil der blive designet omkring dem i designfasen.
De krav der er blevet nået frem til i kravspecifikationen vil blive forklaret i det følgende:
	
\subsection{Funktionelle krav}


\subsubsection{Real-time eye-tracking}
Dette krav er stillet af udbyderen af projektet, og kan
betragtes som det fundamentale krav i projektet. Projektet udspringer fra et tidligere
projekt, hvor billedbehandlingen af kameradata foregik offline. Det skal optimeres i
en tilstrækkelig grad til at kunne køre i realtime i stedet for, med en framerate på
100 billeder per sekund.


\subsubsection{Kalibrering}
For at systemet skal fungere korrekt skal der først foretages en kalibrering.
Man vil godt kunne få data fra systemet, men uden en indledende kalibrering vil man ikke
kunne oversætte de data til et sæt skærmpkoordinater. Denne kalibrering er derfor
et essentielt krav til systemet.


\subsubsection{Output}
resultater af måling skal ende i en log-fil tilgængelig til brugeren. Dette krav
stammer fra et ønske fra udbyder om at have adgang til data efter behandling. Der vil i
design fasen af projektet blive fremsat en protokol, der beskriver hvordan data skal
gemmes i log-filen. 


\subsubsection{Brugertilgang}
Brugergrænsefladen behøver ikke nødvendigvis at være ret optimeret (Den er også lavet i python),
men skal derimod opfylde nogle funktionelle krav, hvilket kan betragtes som en
anden form for optimering, brugeroplevelsesoptimering. Den optimering er beregnet til at
sørge for at applikationen bliver designet og implementeret på en måde som medfører en
bedre oplevelse for brugeren, i form af en simpel brugergrænseflade, tilstrækkelig
funktionalitet til at dække formodede behov, samt et design der gør det let at udvide
applikationen på et senere tidspunkt.

De resterende funktionelle krav er blevet uddybet ved hjælp af Use Cases.


\subsection{Ikke-funktionelle krav}




\subsubsection{Fejlmargin}



\subsubsection{Real-time}
  


\subsection{Overvejelser angående kodesprog}
Det er blevet aftalt med projektudbyder at prototypen skal udarbejdes med en C++ backend og Python frontend.
C++ er valgt som backend fordi det er et relativt hurtigt kodesprog, og fordi gruppen har tidligere erfaring
med C++. Python blev foreslået til gruppen af projektvejleder, og idet Python har en udvidelse der tillader
en let måde at have Python frontend samt C++ backend var det en oplagt mulighed at anvende Python til front
end delen af prototypen.


\end{document}