\documentclass[kravspec.tex]{subfiles}
\begin{document}
\section{Funktionelle krav}
Følgende funktionelle krav for systemet er blevet stillet: \\
\begin{itemize}
	\item 
	Kalibrering: Systemet skal kunne kalibreres således at der kan findes en fornuftig middelværdi for trigger-niveauet. Derudover skal specifikke ukendte variabler kunne kalibreres ved hjælp af interpolation.  
	\item
	Output: Der forventes en fil indeholdende de målte XY-koordinater med tilhørende timestamp, for hver triggersignal. Formatet skal bestemmes.
	\item
	Brugertilgang: Ved hjælp af use case teknikken vil en yderlige række krav blive stillet. Disse vil lægge grundlag for bruger-program-interaktioner. Use-case-kravene er opstillet i afsnit \ref{usec}.  
\end{itemize}

\subsection{Use-cases}	
\label{usec}
\begin{enumerate}
	\item Kalibrering: \\Initierer en række kalibreringer før brug. 
	\item Start måling: \\Igangsætter måling og kreere en tilhørende log-fil.
	\item Pause måling: \\Giver brugeren mulighed for at pause igangværende måling. Herved vil der ikke blive kreeret en ny log-fil. 
	\item Slut måling: \\Afslutter måling.
	\item Gem indstillinger: \\Gemmer en fil med brugerens nuværende indstillinger.
	\item Indlæs indstillinger: \\Indlæser indstillinger fra gemt fil.
	\item Vælg kamera-input: \\ Giver brugeren mulighed for at vælge imellem potentielle kamera-inputs.

\end{enumerate}
	
\end{document}