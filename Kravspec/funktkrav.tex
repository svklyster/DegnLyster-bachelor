\documentclass[kravspec.tex]{subfiles}
\begin{document}
\section{Funktionelle krav}
Følgende funktionelle krav for systemet er blevet stillet: \\
\begin{enumerate}
	\item 
	Kalibrering: Før måling skal programmet kunne kalibreres.
	\item
	Output: Resultater af måling skal ende i en log-fil tilgængelig til brugeren.
	\item
	Brugertilgang: Ved hjælp af use case teknikken vil en yderlige række krav blive stillet. Disse vil lægge grundlag for bruger-program-interaktioner. Use-case-kravene er opstillet i afsnit \ref{usec}.  
\end{enumerate}

\subsection{Kalibrering}
Kalibrering: Specifikke ukendte variabler skal kunne kalibreres ved hjælp af interpolation. Herved skal programmet kunne tilpasses testpersonens fysiske forhold til kameraet. 
\begin{figure}[h]
\centering
\includegraphics[width=0.7\linewidth]{../Kamera-testperson}
\caption{Kameraets position i forhold til testperson}
\label{fig:Camposition}
\end{figure}


Derudover skal programmet kunne kalibreres således at der kan findes tærskler (threshold-values) for trigger-niveauet: En værdi når trigger-niveauet går højt, og en værdi når trigger-niveauet går lavt.

\begin{figure}[H]
\centering
\includegraphics[width=0.7\linewidth]{../Trigger-threshold}
\caption{Eksempel på tærskelværdier for trigger-signalet}
\label{fig:Trigger-threshold}
\end{figure}



\subsection{Output}
For hver igangsat session skal programmet generere en (eller flere) log-fil med følgende data:
\indent \begin{itemize}
	\item Nuværende program-konfiguration
	\item Noter fra bruger
	\item 	Kommasepareret målingsdata med følgende format: \\
	\textit{X-koordinat, Y-koordinat, Samplenummer, Trigger-niveau (0 for lav, 1 for høj)} 
\end{itemize}

\subsection{Use-cases}	
\label{usec}
\begin{enumerate}
	\item Kalibrering: \\Initierer en række kalibreringer før brug. 
	\item Start måling: \\Igangsætter måling og kreere en tilhørende log-fil.
	\item Pause måling: \\Giver brugeren mulighed for at pause igangværende måling. Herved vil der ikke blive kreeret en ny log-fil. 
	\item Slut måling: \\Afslutter måling.
	\item Gem indstillinger: \\Gemmer en fil med brugerens nuværende indstillinger.
	\item Indlæs indstillinger: \\Indlæser indstillinger fra gemt fil.
	\item Vælg kamera-input: \\ Giver brugeren mulighed for at vælge imellem potentielle kamera-inputs.

\end{enumerate}
	
\end{document}