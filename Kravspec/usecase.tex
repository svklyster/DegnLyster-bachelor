\documentclass[kravspec.tex]{subfiles}
\begin{document}
	\subsubsection{Use case 1: Kalibrering}
	\begin{tabular}{|l|p{7.7cm}|}
		\hline \textbf{Sektion} & \textbf{Kommentar} \\ 
		\hline Mål & At kalibrere af systemet \\ 
		\hline Initiering & Initieres af aktøren \textit{bruger} \\ 
		\hline Aktører & Aktøren \textit{bruger} og aktøren \textit{kamera} \\ 
		\hline Antal samtidige forekomster & 1 \\ 
		\hline Startbetingelser & Computerprogrammet skal være opstartet, kameraret skal være tændt og tilsluttet \\ 	
		\hline Slutresultat - succes & Computerprogrammet har modtaget korrekt mængde data til kalibrering. \\ 
		\hline Slutresultat - undtagelse &  Ingen undtagelse. \\ 
		\hline Normal forløb &  \\ 
		\hline Undtagelsesforløb & Ingen \\ 
		\hline 
	\end{tabular} \\ \\
	Kommentar: Når kalibreringsprocessen er gennemført vil programmet bruge det... blabla
	
	\subsubsection{Use case 2: Start måling}
	\begin{tabular}{|l|p{7.7cm}|}
		\hline \textbf{Sektion} 	& \textbf{Kommentar} \\ 
		\hline Mål  & Programmet påbegynder real-time eye-tracking \\ 
		\hline Initiering  & Initieres af aktøren \textit{bruger} \\ 
		\hline Aktører & Aktøren \textit{bruger} og aktøren \textit{kamera} \\ 
		\hline Antal samtidige forekomster & 1 \\ 
		\hline Startbetingelser & Computerprogrammet skal være opstartet, kameraret skal være tændt, programmet skal være kalibreret  \\ 
		\hline Slutresultat – succes & Programmet har påbegyndt real-time eye-tracking \\ 
		\hline Slutresultat – undtagelse & Programmet alarmerer bruger at der ikke er foretaget kalibrering \\ 
		\hline Normal forløb &  \\ 
		\hline Undtagelsesforløb &  \\ 
		\hline 
	\end{tabular}

	\subsubsection{Use case 5: Gem indstillinger}
	\begin{tabular}{|l|p{7.7cm}|}
		\hline \textbf{Sektion} 	& \textbf{Kommentar} \\ 
		\hline Mål  & System gemmer nuværende indstillinger i brugerdefineret fil \\ 
		\hline Initiering  & Initieres af aktøren \textit{bruger} \\ 
		\hline Aktører & Aktøren \textit{bruger} \\ 
		\hline Antal samtidige forekomster & 1 \\ 
		\hline Startbetingelser & Computerprogrammet skal være opstartet og indstillinger skal være ændret  \\ 
		\hline Slutresultat – succes & Systemet gemmer nuværende indstillinger i den valgte fil \\ 
		\hline Slutresultat – undtagelse & Programmet alarmerer bruger at indstillinger ikke kunne gemmes \\ 
		\hline Normal forløb & Bruger vælger filnavn og placering af den fil hvor indstillinger skal gemmes \\ 
		\hline Undtagelsesforløb & Bruger bliver returneret til menuen efter alarmering \\ 
		\hline 
	\end{tabular}
	
	\subsubsection{Use case 6: Indlæs indstillinger}
	\begin{tabular}{|l|p{7.7cm}|}
		\hline \textbf{Sektion} 	& \textbf{Kommentar} \\ 
		\hline Mål  & System anvender indstillinger fra fil \\ 
		\hline Initiering  & Initieres af aktøren \textit{bruger} \\ 
		\hline Aktører & Aktøren \textit{bruger} \\ 
		\hline Antal samtidige forekomster & 1 \\ 
		\hline Startbetingelser & Computerprogrammet skal være opstartet og en gyldig indstillings-fil skal findes  \\ 
		\hline Slutresultat – succes & Systemet anvender indstillinger fra den valgte fil \\ 
		\hline Slutresultat – undtagelse & Programmet alarmerer bruger at indstillinger ikke kunne indhentes \\ 
		\hline Normal forløb & Bruger vælger indstillings-fil der skal anvendes \\ 
		\hline Undtagelsesforløb & Bruger bliver returneret til menuen efter alarmering \\ 
		\hline 
	\end{tabular}
	
	\subsubsection{Use case 7: Vælg kamera-input}
	\begin{tabular}{|l|p{7.7cm}|}
		\hline \textbf{Sektion} 	& \textbf{Kommentar} \\ 
		\hline Mål  & System anvender valgt kamera \\ 
		\hline Initiering  & Initieres af aktøren \textit{bruger} \\ 
		\hline Aktører & Aktøren \textit{bruger} og aktøren \textit{kamera} \\ 
		\hline Antal samtidige forekomster & 1 \\ 
		\hline Startbetingelser & Computerprogrammet skal være opstartet og kameraret skal være tændt  \\ 
		\hline Slutresultat – succes & Systemet anvender det valgte kamera \\ 
		\hline Slutresultat – undtagelse & Programmet alarmerer bruger at der ikke er valgt kamera \\ 
		\hline Normal forløb & \begin{enumerate}
			\item Bruger klikker på "Camera Input".
			\item Bruger vælger kamera der skal anvendes fra liste.
			\item Bruger godkender eller afviser valg af kamera.
			\begin{enumerate}
			\item Bruger godkendte kamera. Valgt kamera bliver sat som input, og bruger returneres til main GUI. Use Case afsluttes.
			\item Bruger afviser valg af kamera. Bruger returneres til liste (Punkt 2).
			\end{enumerate}
		\end{enumerate} \\ 
		\hline Undtagelsesforløb & Bruger bliver returneret til menuen efter alarmering \\ 
		\hline 
	\end{tabular}
	
	\subsubsection{Use case 8: Aktiver "Raw" data logging}
	\begin{tabular}{|l|p{7.7cm}|}
		\hline \textbf{Sektion} 	& \textbf{Kommentar} \\ 
		\hline Mål  & System logger rå video data ved måling (Use Case 2) \\ 
		\hline Initiering  & Initieres af aktøren \textit{bruger} \\ 
		\hline Aktører & Aktøren \textit{bruger} \\ 
		\hline Antal samtidige forekomster & 1 \\ 
		\hline Startbetingelser & Computerprogrammet skal være opstartet \\ 
		\hline Slutresultat – succes & Programmet viser filnavn, er klar tilrå video data logging, og returnerer til idle. \\ 
		\hline Slutresultat – undtagelse & Programmet viser \textit{ikke} et filnavn og returnerer til idle. \\ 
		\hline Normal forløb & \begin{enumerate}
			\item Bruger klikker på "Raw Data Logging" knappen.
			\item Bruger vælger sti og filnavn.
			\item Program viser filnavn, og returnerer til idle. Use case afsluttes.
		\end{enumerate} \\ 
		\hline Undtagelsesforløb & Bruger bliver returneret til menuen efter alarmering \\ 
		\hline 
	\end{tabular}
	
\end{document}