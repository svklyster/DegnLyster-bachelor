\documentclass[kravspec.tex]{subfiles}
\begin{document}
		\subsubsection{Use case 1: Opret session}
		\begin{tabular}{|l|p{7.7cm}|}
			\hline \textbf{Sektion} & \textbf{Kommentar} \\ 
			\hline Mål & At oprette en ny session til real-time eye-tracking \\ 
			\hline Initiering & Initieres af aktøren \textit{bruger} \\ 
			\hline Aktører & Aktøren \textit{bruger} og aktøren \textit{kamera}\\ 
			\hline Antal samtidige forekomster & 1 \\ 
			\hline Startbetingelser & Computerprogrammet skal være opstartet, kameraret skal være tændt og tilsluttet \\ 	
			\hline Slutresultat - succes & En ny session til real-time eye-tracking er blevet oprettet. \\ 
			\hline Slutresultat - undtagelse &  Ny session er ikke blevet oprettet. \\ 
			\hline Normal forløb & \begin{enumerate}
				\item \textit{Bruger} klikker på ”Create session”
				\item Programmet beder \textit{bruger} angive en fil-sti hvor sessiones filer skal oprettes.
				\item Programmet beder \textit{bruger} vælge imellem tilgængelige kamera-input.
				\item Programmet beder \textit{bruger} vælge om den rå data fra \textit{kamera} skal gemmes.
				
				
				\item Programmet giver \textit{bruger} mulighed for at skrive eventuelle noter til sessionen i et tekst-felt. 
				\item Programmet opretter en data-fil og en opsætnings-fil.
				\item \textit{Bruger} bliver returneret til menu med beskeden ”Session created”.
				
			\end{enumerate} \\ 
			\hline Undtagelsesforløb & Programmet kan ikke oprette de ønskede filer i den angivne fil-sti. Programmet returnere \textit{bruger} til menu med fejlmeddelelse fra systemet.  \\ 
			\hline 
		\end{tabular} \\ \\
		
	\subsubsection{Use case 2: Kalibrering}
	\begin{tabular}{|l|p{7.7cm}|}
		\hline \textbf{Sektion} & \textbf{Kommentar} \\ 
		\hline Mål & At kalibrere systemet \\ 
		\hline Initiering & Initieres af aktøren \textit{bruger} \\ 
		\hline Aktører & Aktøren \textit{bruger} og aktøren \textit{kamera} \\ 
		\hline Antal samtidige forekomster & 1 \\ 
		\hline Startbetingelser & Computerprogrammet skal være opstartet, kameraret skal være tændt og tilsluttet, gyldig session skal være oprettet. \\ 	
		\hline Slutresultat - succes & Computerprogrammet har modtaget data til kalibrering. \\ 
		\hline Slutresultat - undtagelse &  Kalibrering annulleret. \\ 
		\hline Normal forløb & \begin{enumerate}
			\item \textit{Bruger} klikker på ”Calibration”
			\item Programmet spørger om bruger ønsker at foretage kalibrering.
			\begin{enumerate}
				\item \textit{Bruger} klikker på ”Cancel”.
				Use case afbrydes.
				Se undtagelsesforløb punkt 2 for denne use case.
				\item \textit{Bruger} klikker på ”Continue”. 
				Use case fortsættes i punkt 3.
			\end{enumerate}
			\item \textit{Bruger} bliver bedt om at gennemføre kalibreringsrutine.
			\item \textit{Bruger} bliver returneret til menu med beskeden ”Calibration complete”.

		\end{enumerate} \\ 
		\hline Undtagelsesforløb & \begin{enumerate}
			\item Hvis der ikke er oprettet en session informeres \textit{Bruger} om dette.
			\item \textit{Bruger} bliver returneret til menu.
		\end{enumerate}  \\ 
		\hline 
	\end{tabular} \\ \\
	
	\subsubsection{Use case 3: Start måling}
	\begin{tabular}{|l|p{7.7cm}|}
		\hline \textbf{Sektion} 	& \textbf{Kommentar} \\ 
		\hline Mål  & Programmet påbegynder real-time eye-tracking \\ 
		\hline Initiering  & Initieres af aktøren \textit{bruger} \\ 
		\hline Aktører & Aktøren \textit{bruger} og aktøren \textit{kamera} \\ 
		\hline Antal samtidige forekomster & 1 \\ 
		\hline Startbetingelser & Computerprogrammet skal være opstartet, \textit{kamera} skal være tændt, programmet skal være kalibreret. \\ 
		\hline Slutresultat – succes & Programmet har påbegyndt real-time eye-tracking\\ 
		\hline Slutresultat – undtagelse & Programmet alarmerer \textit{bruger} at der ikke er foretaget kalibrering \\ 
		\hline Normal forløb & \begin{enumerate}
			\item \textit{Bruger} klikker på knappen ”Start”.
			\item Programmet starter ny måling.
			\item Visuel feedback på GUI viser at måling er i gang.
		\end{enumerate} \\  
		\hline Undtagelsesforløb & Programmet kan ikke starte ny måling. Programmet melder
			at kalibrering ikke er foretaget.\\
		\hline 
	\end{tabular}

	\subsubsection{Use case 4: Stop måling}
		\begin{tabular}{|l|p{7.7cm}|}
			\hline \textbf{Sektion} 	& \textbf{Kommentar} \\ 
			\hline Mål  & Progammet stopper real-time eye-tracking \\ 
			\hline Initiering  & Initieres af aktøren \textit{bruger} \\ 
			\hline Aktører & Aktøren \textit{bruger} og aktøren \textit{kamera} \\ 
			\hline Antal samtidige forekomster & 1 \\ 
			\hline Startbetingelser & En real-time eye-tracking måling  skal køre.   \\ 
			\hline Slutresultat – succes & Programmet stopper nuværende real-time eye-tracking.\\ 
			\hline Slutresultat – undtagelse & Ingen undtagelse. \\ 
			\hline Normal forløb & \begin{enumerate}
				\item \textit{Bruger} klikker på knappen ”Stop”.
				\item Visuel feedback på GUI viser at målingen er stoppet. Målingen bliver afsluttet. 
			\end{enumerate} \\ 
			\hline Undtagelsesforløb & Intet undtagelsesforløb \\ 
			\hline 
		\end{tabular}

	\subsubsection{Use case 5: Gem indstillinger}
	\begin{tabular}{|l|p{7.7cm}|}
		\hline \textbf{Sektion} 	& \textbf{Kommentar} \\ 
		\hline Mål  & System gemmer nuværende indstillinger i brugerdefineret fil \\ 
		\hline Initiering  & Initieres af aktøren \textit{bruger} \\ 
		\hline Aktører & Aktøren \textit{bruger} \\ 
		\hline Antal samtidige forekomster & 1 \\ 
		\hline Startbetingelser & Computerprogrammet skal være opstartet \\ 
		\hline Slutresultat – succes & Systemet gemmer nuværende indstillinger i den valgte fil \\ 
		\hline Slutresultat – undtagelse & Programmet alarmerer \textit{bruger} at indstillinger ikke kunne gemmes \\ 
		\hline Normal forløb & \begin{enumerate}
			\item \textit{Bruger} klikker på "Save preferences"
			\item \textit{Bruger} vælger filnavn og placering af den fil hvor indstillinger skal gemmes
		\end{enumerate} \\ 
		\hline Undtagelsesforløb & \textit{Bruger} bliver returneret til menuen efter alarmering \\ 
		\hline 
	\end{tabular}
	
	\subsubsection{Use case 6: Indlæs indstillinger}
	\begin{tabular}{|l|p{7.7cm}|}
		\hline \textbf{Sektion} 	& \textbf{Kommentar} \\ 
		\hline Mål  & System anvender indstillinger fra fil \\ 
		\hline Initiering  & Initieres af aktøren \textit{bruger} \\ 
		\hline Aktører & Aktøren \textit{bruger} \\ 
		\hline Antal samtidige forekomster & 1 \\ 
		\hline Startbetingelser & Computerprogrammet skal være opstartet og en gyldig indstillings-fil skal findes  \\ 
		\hline Slutresultat – succes & Systemet anvender indstillinger fra den valgte fil \\ 
		\hline Slutresultat – undtagelse & Programmet alarmerer \textit{bruger} at indstillinger ikke kunne indhentes \\ 
		\hline Normal forløb & \begin{enumerate}
			\item \textit{Bruger} klikker på "Load preferences"
			\item \textit{Bruger} vælger indstillings-fil der skal anvendes
		\end{enumerate} \\ 
		\hline Undtagelsesforløb & \textit{Bruger} bliver returneret til menuen efter alarmering \\ 
		\hline 
	\end{tabular}
	
	\subsubsection{Use case 7: Indlæs rå data}
	\begin{tabular}{|l|p{7.7cm}|}
		\hline \textbf{Sektion} 	& \textbf{Kommentar} \\ 
		\hline Mål  & System indlæser rå data fra tidligere session \\ 
		\hline Initiering  & Initieres af aktøren \textit{bruger} \\ 
		\hline Aktører & Aktøren \textit{bruger} \\ 
		\hline Antal samtidige forekomster & 1 \\ 
		\hline Startbetingelser & Computerprogrammet skal være opstartet og en gyldig fil-sti med data fra en session skal findes  \\ 
		\hline Slutresultat – succes & Systemet indlæser rå data fra korrekte data-filer i tidligere session \\ 
		\hline Slutresultat – undtagelse & Programmet alarmerer \textit{bruger} at de korrekte data-filer ikke kunne indhændtes \\ 
		\hline Normal forløb & \begin{enumerate}
			\item \textit{Bruger} klikker på knappen "Get raw data"
			\item \textit{Bruger} vælger fil-sti til tidligere session hvor data-filer med rå data findes
			\item Programmet indlæser valgt data
			\item \textit{Bruger} bliver returneret til menu
			\end{enumerate} \\
		\hline Undtagelsesforløb & \textit{Bruger} bliver returneret til menuen efter alarmering \\ 
		\hline 
	\end{tabular}
	
\end{document}