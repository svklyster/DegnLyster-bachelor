\documentclass[softwarearkitektur.tex]{subfiles}
\begin{document}
\section{Indledning}
For at oprette en software arkitektur der overholder de krav stillet i kravspecifikationen, er der gjort brug af UML (Unified Modelling Language) \cite{UMLurl}. UML tillader en tilnærmelsesvis direkte omskrivning af krav opstillet som use cases, til UML-diagrammer der skitserer en software arkitektur. 
Da der i kravspecifikationen er gjort brug af UML til udarbejdelse af de forskellige use cases, har det derfor været muligt af skrive sekvensdiagrammer ud fra hver enkelte use case. 

Sideløbende med udviklingen af sekvensdiagrammerne er de forskellige klasser blevet forfattet. Følgende afsnit beskriver grundlæggende tanker og argumentation for valget af klasser.

\end{document}